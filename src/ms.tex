%%%%%%%%%%%%%%%%%%%%%%%%%%%%%%%%%%%%%%%%%%%%%%%%%%%%%%%%%%%%%%%%%%%%%%%%%%%%%%%
%
%   Please read the CONTRIBUTING.md file in this repository for notes about
%   style and latex conventions!
%
%%%%%%%%%%%%%%%%%%%%%%%%%%%%%%%%%%%%%%%%%%%%%%%%%%%%%%%%%%%%%%%%%%%%%%%%%%%%%%%

\documentclass[modern]{aastex631}
\usepackage{xspace}
\usepackage[utf8]{inputenc}
\usepackage[T1]{fontenc}
\usepackage{ulem}

% To allow putting figures in a subdir
\graphicspath{{figures/}}

\submitjournal{ApJ}

\shorttitle{Astropy Project III}
\shortauthors{The Astropy Collaboration}

\newcommand{\escapecmd}[1]{\texttt{\detokenize{#1}}}

% Packages / projects / programming - for consistency!
\newcommand{\package}[1]{\texttt{#1}\xspace}
\newcommand{\github}{\package{GitHub}}
\newcommand{\python}{\package{Python}}
\newcommand{\astropy}{Astropy\xspace}
\newcommand{\astropypkg}{\package{astropy}}

% For consistency:
\newcommand{\sectionname}{Section\xspace}
\renewcommand{\figurename}{Figure\xspace}
\newcommand{\equationname}{Equation\xspace}
\renewcommand{\tablename}{Table\xspace}

% Words that should not be hyphenated
\hyphenation{NumFOCUS}

% For commenting - can be deleted before submission
\usepackage[colorinlistoftodos]{todonotes}
\newcommand{\inlinecomment}[2]{\todo[inline]{#1: #2}\xspace}
\newcommand{\comment}[2]{\todo{#1: #2}\xspace}

\usepackage{newunicodechar,graphicx}
\DeclareRobustCommand{\okina}{%
 \raisebox{\dimexpr\fontcharht\font`A-\height}{%
 \scalebox{0.8}{`}%
 }%
}
\newunicodechar{ʻ}{\okina}


% Links to documentation: subpackages
\newcommand{\astropysubpkg}[1]{\href{http://docs.astropy.org/en/stable/#1/index.html}{\texttt{astropy.#1}}\xspace}
\newcommand{\astropyiosubpkg}[1]{\href{http://docs.astropy.org/en/stable/io/#1/index.html}{\texttt{astropy.io.#1}}\xspace}
\newcommand{\astropywcsaxes}{\href{http://docs.astropy.org/en/stable/visualization/wcsaxes/index.html}{\texttt{astropy.visualization.wcsaxes}}\xspace}

% Links to documentation: classes
\newcommand{\astropyskycoord}{\href{http://docs.astropy.org/en/stable/api/astropy.coordinates.SkyCoord.html}{\texttt{SkyCoord}}\xspace}
\newcommand{\astropyQuantity}{\href{http://docs.astropy.org/en/stable/api/astropy.units.Quantity.html}{\texttt{Quantity}}\xspace}
\newcommand{\astropyTime}{\href{http://docs.astropy.org/en/stable/api/astropy.time.Time.html}{\texttt{Time}}\xspace}
\newcommand{\astropyTable}{\href{http://docs.astropy.org/en/stable/api/astropy.table.Table.html}{\texttt{Table}}\xspace}

\newcommand{\secauthor}[1]{{\color{blue}Author:~\textit{#1}}}
\newcommand{\secunfilled}{{\color{red}Author:~\textit{Looking for volunteers!}}}

\begin{document}

\draft{\today}

\title{The Astropy Project: \\
       Sustaining and Growing a Community-oriented Open-source Project and
       the Latest Major Release (v5.0) of the Core Package}

\correspondingauthor{Astropy Coordination Committee}
\email{coordinators@astropy.org}

\author{Astropy Collaboration}
\noaffiliation
{\let\thefootnote\relax\footnote{{The author list has two parts: the authors that made significant contributions to the writing of the paper in order of contribution, followed by contributors to the \astropy Project in alphabetical order. \textbf{The position in the author list does not correspond to contributions to the \astropy Project as a whole.} A more complete list of contributors to the core package can be found in the \href{https://github.com/astropy/astropy/graphs/contributors}{package repository}, and at the \href{http://www.astropy.org/team.html}{\astropy team webpage}.}}}

% \input{author.tex}

\begin{abstract}
The \astropy Project supports and fosters the development of open-source and openly-developed
\python packages that provide commonly-needed functionality to the astronomical
community.
A key element of the \astropy Project is the core package \astropypkg, which serves as the
foundation for more specialized projects and packages.
In this article, we summarize key features in the core package as of the recent major
release, version 5.0, and provide major updates for the project.
We then discuss supporting a broader ecosystem of inter-operable packages,
including connections with several astronomical observatories and missions.
We also revisit the future outlook of the \astropy Project and the current
status of Learn Astropy. 
We conclude by raising and discussing the current and future challenges facing the project.
\end{abstract}

\keywords{%
    Astrophysics - Instrumentation and Methods for Astrophysics
    ---
    methods: data analysis
    ---
    methods: miscellaneous
}


\section{Introduction} \label{sec:intro}

\secauthor{Adrian Price-Whelan}

The \python programming language is a high-level, interpreted (as opposed to
compiled) programming language that has become an industry standard across many
computational domains, technological sectors, and fields of research.
Despite claims to the contrary \citep{Portegies-Zwart:2020}, \python enables
scalable, time- and energy-efficient code execution \citep[e.g.,][]{Augier:2021}
with a focus on code readability, ease of use, and interoperability with other
languages.
Over the last decade, \python has grown enormously in popularity to become a
dominant programming language in the astronomical and broader scientific
communities.
For example, Figure~\ref{fig:python-mentions} shows the number of yearly
full-text mentions of \python as compared to a few other programming languages
in refereed articles in the astronomical literature, demonstrating its nearly
exponential growth in popularity.
The rapid adoption of \python by astronomy researchers, students, observatories,
and technical staff combined with an associated increase in awareness and
interest about open-source software tools is contributing to a paradigm shift in
the way research is done, data is analyzed, and results are shared in astronomy
and beyond.

One of the factors that has led to its rapid ascent in popularity in scientific
contexts has been the significant, volunteer-driven effort behind developing
community-oriented open-source software tools and fostering communities of users
and developers that have grown around these efforts.
Today, a broad and feature-diverse ``ecosystem'' of packages exists in the
\python scientific computing landscape: Roughly ordered from general-use to
domain-specific, this landscape now includes packages that provide core
numerical analysis functionality like \package{numpy} \citep{numpy:nature} and
\package{scipy} \citep{scipy}, visualization frameworks like
\package{matplotlib} \citep{matplotlib}, machine learning and data analysis
packages like \package{tensorflow} \citep{tensorflow}, \package{pymc3}
\citep{Salvatier:2016}, and \package{emcee} \citep{emcee}, domain-specific
libraries like \package{yt} \citep{yt:2011}, \package{plasmapy}
\citep{plasmapy}, \package{sunpy} \citep{sunpy:apj}, \package{Biopython}
\citep{biopython}, and \package{sympy} \citep{sympy} (to name a few in each
category).
The \astropypkg \citep{astropy:2013, astropy:2018} core package began in this
vein, as an effort to consolidate the development of commonly-used functionality
needed to perform astronomical research into a community-developed \python
package.

The \astropypkg core package was one of the first large, open-source \python
packages developed for astronomy and provides, among other things, software
functionality for reading and writing astronomy-specific data formats (e.g.,
FITS), transforming and representing astronomical coordinates, and representing
and propagating physical units in code.
An early description of the core functionality in \astropypkg can be found in
the first Astropy paper \citep{astropy:2013} or in detail in the core package
documentation.\footnote{\url{https://docs.astropy.org/}}
The \astropypkg core package is now largely stable in that the software
interface does not change without sufficient and significant motivation, and the
addition of new features into the core package has slowed significantly as
compared to the first years of its development.
This is largely driven by the fact that the core package now represents just one
piece of the broader astronomy \python context, and much new feature development
is now happening in more specialized packages that are expanding the
capabilities of the Astropy ecosystem by building on top of the foundations laid
by the \astropypkg core package.
Because of this natural expansion, the name Astropy has grown in scope beyond a
single \python library to become ``the Astropy Project.''

The Astropy Project is a community effort that represents the union of the
\astropypkg core package, the ecosystem of astronomy-specific software tools
that are interoperable with \astropypkg (Astropy Affiliated Packages),
\emph{and} the community of users, developers, and maintainers that participate
in Astropy efforts.
However, there is no institution responsible for managing the Astropy Project,
for funding or maintaining its development, or sustaining it into the future:
The Project is maintained and coordinated largely by volunteers.
While new Astropy-affiliated packages are being developed that expand upon the
core functionality in the \astropypkg package, representing a natural expansion
of the Astropy Project ecosystem, the needs of and challenges faced by the
Project are evolving.
In particular, the transition from focusing our energy on development and
maintenance of a single core package, to instead sustaining the core package and
fostering the development of the community and its expansion has been a key
issue faced by the Astropy Project in the last several years.

In this Article, we briefly describe recent key updates in the \astropypkg core
package since the last Astropy paper (``Paper II''; \citealt{astropy:2018}),
major updates in the governance, contributor base, and funding of the Project,
and discuss some of the future plans and challenges faced by the Astropy
Project.


\begin{figure}
    \begin{centering}
        \includegraphics[width=\textwidth]{figures/python-mentions.pdf}
        \caption{
            Yearly full-text mentions of programming languages (indicated in the
            figure legend) in refereed publications in the astronomical
            literature database in the Astrophysics Data System (ADS;
            \citealt{ads}).
            \python has rapidly become the dominant programming language
            mentioned in refereed articles over the last 10 years.
        }
        \label{fig:python-mentions}
    \end{centering}
\end{figure}


\section{Major Updates to the Astropy Core Package} \label{sec:core-updates}

\subsection{New Long-term Support (LTS) Version: v5.0} \label{sec:core-v50}

\secauthor{Tom Robitaille}

Summarize a few notable things in v5.0 as compared to the last LTS release.


\subsection{Highlighted Feature Development} \label{sec:core-features}

\secauthor{Nathaniel Starkman, Marten van Kerkwijk}

Summarize major pieces of development since the last paper (since v2.0). A few
ideas (taken from What's New pages) below, but feel free to expand on or remove
from this list:
\begin{itemize}
    \item Support for representing and transforming velocity data in coordinates, epoch propagation (v3.0)
    \item Improved support for astronomical time series: TimeSeries object (v3.2), Box Least Squares periodogram (v3.1)
    \item Overall improved support of Quantity throughout numpy (v4.0) and scipy
    \item Native support for Time, Quantity, and SkyCoord objects in Astropy tables
    \item TODO: something about WCS, SpectralCoord?
\end{itemize}


\section{Major Updates in the Astropy Project} \label{sec:project-updates}

\subsection{Project governance} \label{sec:project-governance}

\secauthor{Erik Tollerud}

Briefly summarize new procedures and governance structure, new CoCo, election
process overview, etc.

\subsection{Contributor base} \label{sec:project-contributors}

\secauthor{Adrian Price-Whelan}

Overview and statistics of contributors. Highlight changes since v2.0.

\begin{figure}
    \begin{centering}
        \includegraphics{figures/contributor-summary.pdf}
        \caption{Placeholder figure!}
        \label{fig:contributor-summary}
    \end{centering}
\end{figure}

% Transition of needs: From new features, to community development, sustainability, ...
% - pipeline between User, Participant, Code contributor, Maintainer, Coordinator.

\subsection{Inclusion, Diversity, and Equity Programs} \label{sec:project-ide}

\secauthor{Lía Corrales}


\subsection{Funding} \label{sec:project-funding}

\secauthor{Aarya Patil}

Summarize funding sources (Moore, NASA) and amounts and what this has been used
for.


\section{Supporting the Ecosystem of Astronomical Python Software}
\label{sec:ecosystem}

\subsection{Community-oriented infrastructure}

\secauthor{Nicholas Earl}

Infrastructure packages that exist to support and help others maintain their
package infrastructure.

This includes an overview of the sphinx-astropy tools, the pytest extensions
maintained by astropy, and the package template.

\subsection{Affiliated packages}

\secauthor{Matt Craig, Brett Morris}

Highlight a few new affiliated packages and major updates to existing ones.
Include a big table of all affiliated packages and references (as in v2.0
paper).

All affiliated packages are listed on the astropy 
website\footnote{\url{https://www.astropy.org/affiliated}}. Since 
\cite{astropy:2018}, there has been an explosion of gravitational-wave affiliated 
packages including: \texttt{PyCBC} for exploring GW sources, lenstronomy for 
modeling strong gravitational lenses, \texttt{ligo.skymap} for visualizing GW 
probability maps, and \texttt{EinsteinPy} for general relativity and gravitational astronomy. 
There have been several packages related to HEALPix: \texttt{astropy-healpix}
for a BSD licensed HEALPix implementation, and \texttt{mocpy} for Multi-Order 
Coverage maps. \texttt{astroalign} as been introduced for astrometric registration, 
and \texttt{python-cpl} has been added for ESO pipelines and VLT data products. 
For ground-based astronomy, \texttt{baseband} has added IO capabilities for 
VLBI, and \texttt{SpectraPy} brings slit spectroscopy to the astropy ecosystem. 
We have \texttt{agnpy} for AGN astrophysics, and \texttt{statmorph} for fitting galactic 
morphological diagnostics. \texttt{saba} gives an interface to sherpa's fitting 
routines. For modeling interstellar dust extinction we have 
\texttt{dust_extinction}, and for extracting features from time series we 
have \texttt{feets}. We also have \texttt{corral} for managing data intensive parallel 
pipelines. \texttt{BayesicFitting} provides an interface for generic Bayesian 
inference, and \texttt{sbpy} does calculations for asteroid and cometary 
astrophysics. Finally, \texttt{synphot} provides an interface for synthetic photometry.

Several of the affiliated packages described in [reference to paper 2] have
had substantial improvements. \texttt{astroquery} \cite{astroquery} has added
access to roughly a dozen new missions and data services, including JWST, and
the project has switched to a continuous release model. Every time a change
is committed to the main development branch it is published on the Python
Package Index (PyPI, ref) and available for installation. Formal releases are
still done a couple of times per year. \texttt{photutils} \cite
{photutils} released its first stable version, indicating that the API will
change less frequently, and there have been several significant performance
improvements.  \texttt{ccdproc} \cite{ccdproc} also released a new major
version, bringing better performance to some image combination operations.
The \texttt{regions} package \cite{pyregions}, for manipulating ds9-style
region definitions \cite{ds9}, added new ways to manipulate regions and
introduced new region types. \texttt{reproject}’s \cite{reproject} major new
feature is a function to align and co-add images to create a mosaic; better
support for parallelization was also added. \texttt{specutils}, the package
that defines containers for 1D and 2D spectra \cite{specutils}, also had its
first stable release and the addition of classes to read JWST data.

\subsection{Connections with data archives}

\secauthor{Adam Ginsburg}

Summary of Astroquery and engagement with archives. Cite astroquery paper.

\subsection{Connections with Observatories and Missions}

Brief summary of efforts in observatory- or mission-driven development that have
contributed to Astropy, and vice versa.

JWST
\secauthor{Larry Bradley}

Gemini
\secunfilled

Cherenkov Telescope Array
\secauthor{Axel Donath, Max Noethe}

Rubin Observatory
\secunfilled

LIGO/Virgo/KAGRA
\secauthor{Leo Singer}
\section{Future Plans for the Astropy Project} \label{sec:future}

\subsection{Roadmap}

\secauthor{Clara Brasseur}

Overview of roadmap and some highlights.

\section{Learn Astropy} \label{sec:learn}

\secauthor{Lía Corrales, David Shupe + Learn team}

\begin{itemize}
\item Current status and scope
\item Vision for the future (review what we said in v2.0 paper)
\item User forums and engagement with user base
\item Workshops
\end{itemize}


\subsection{Current and Future Challenges}

\secunfilled

\begin{itemize}
\item Attracting new contributors when the code has become quite complex,
\item Contributor to maintainer mentoring,
\item Long-term / sustained funding for maintaining infrastructure,
\item ...
\end{itemize}


\begin{acknowledgments}

We would like to thank the members of the community that have contributed to
\astropy, that have opened issues and provided feedback, and have supported the
project in a number of different ways.

The \astropy community is supported by and makes use
of a number of organizations and services outside the traditional
academic community. We thank Google for financing and organizing the
Google Summer of Code (GSoC) program, that has funded severals
students per year to work on \astropy related projects over the
summer. These students often turn into long-term contributors. We also
thank NumFOCUS and the Python Software Foundation for financial
support. Within the academic community, we thank
institutions that make it possible that astronomers and other developers on
their staff can contribute their time to the development of
\astropy projects.  We would like acknowledge the support of the
Space Telescope Science Institute, Harvard--Smithsonian Center for Astrophysics,
and the South African Astronomical Observatory.

Furthermore, the \astropy packages would not exist in their current form without
a number of web services for code hosting, continuous integration, and
documentation; in particular, \astropy heavily relies on GitHub, Travis CI,
Appveyor, CircleCI, and Read the Docs.

\astropypkg interfaces with the SIMBAD database, operated at CDS, Strasbourg,
France. It also makes use of the ERFA library \citep{erfa}, which in turn
derives from the IAU SOFA Collection\footnote{\url{http://www.iausofa.org}}
developed by the International Astronomical Union Standards of Fundamental
Astronomy \citep{sofa}.

\end{acknowledgments}

\software{\package{astropy} \citep{astropy:2013, astropy:2018},
          \package{numpy} \citep{numpy:nature},
          \package{scipy} \citep{scipy},
          \package{matplotlib} \citep{matplotlib},
          \package{Cython} \citep{cython}.
        %   \package{SOFA} \citep{sofa},
        %   \package{ERFA} (\citealt{erfa})
          }

\bibliographystyle{aasjournal}
\bibliography{refs}

% \appendix

% \section{List of Affiliated Packages}

% \begin{longrotatetable}
%     \begin{deluxetable*}{cccp{3in}c}
%     \tablecaption{Registry of affiliated packages.}
%     \label{tab:registry}
%     \tablehead{
%         \colhead{Package Name} &
%         \colhead{Stable} &
%         \colhead{PyPI Name} &
%         \colhead{Maintainer} &
%         \colhead{Citation}
%       }
%       \startdata
%         \input{registry.tex}
%       \enddata
%   \end{deluxetable*}
% \end{longrotatetable}


\end{document}
