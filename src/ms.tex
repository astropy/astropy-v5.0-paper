%%%%%%%%%%%%%%%%%%%%%%%%%%%%%%%%%%%%%%%%%%%%%%%%%%%%%%%%%%%%%%%%%%%%%%%%%%%%%%%
%
%   Please read the CONTRIBUTING.md file in this repository for notes about
%   style and latex conventions!
%
%%%%%%%%%%%%%%%%%%%%%%%%%%%%%%%%%%%%%%%%%%%%%%%%%%%%%%%%%%%%%%%%%%%%%%%%%%%%%%%

\documentclass[modern]{aastex631}
\usepackage{xspace}
\usepackage[utf8]{inputenc}
\usepackage[T1]{fontenc}
\usepackage{ulem}

% To allow putting figures in a subdir
\graphicspath{{figures/}}

\submitjournal{ApJ}

\shorttitle{Astropy Project III}
\shortauthors{The Astropy Collaboration}

\newcommand{\escapecmd}[1]{\texttt{\detokenize{#1}}}

% Packages / projects / programming - for consistency!
\newcommand{\package}[1]{\texttt{#1}\xspace}
\newcommand{\github}{\package{GitHub}}
\newcommand{\python}{\package{Python}}
\newcommand{\astropy}{Astropy\xspace}
\newcommand{\astropypkg}{\package{astropy}}

% For consistency:
\newcommand{\sectionname}{Section\xspace}
\renewcommand{\figurename}{Figure\xspace}
\newcommand{\equationname}{Equation\xspace}
\renewcommand{\tablename}{Table\xspace}

% Words that should not be hyphenated
\hyphenation{NumFOCUS}

% For commenting - can be deleted before submission
\usepackage[colorinlistoftodos]{todonotes}
\newcommand{\inlinecomment}[2]{\todo[inline]{#1: #2}\xspace}
\newcommand{\comment}[2]{\todo{#1: #2}\xspace}

\usepackage{newunicodechar,graphicx}
\DeclareRobustCommand{\okina}{%
 \raisebox{\dimexpr\fontcharht\font`A-\height}{%
 \scalebox{0.8}{`}%
 }%
}
\newunicodechar{ʻ}{\okina}


% Links to documentation: subpackages
\newcommand{\astropysubpkg}[1]{\href{http://docs.astropy.org/en/stable/#1/index.html}{\texttt{astropy.#1}}\xspace}
\newcommand{\astropyiosubpkg}[1]{\href{http://docs.astropy.org/en/stable/io/#1/index.html}{\texttt{astropy.io.#1}}\xspace}
\newcommand{\astropywcsaxes}{\href{http://docs.astropy.org/en/stable/visualization/wcsaxes/index.html}{\texttt{astropy.visualization.wcsaxes}}\xspace}

% Links to documentation: classes
\newcommand{\astropyskycoord}{\href{http://docs.astropy.org/en/stable/api/astropy.coordinates.SkyCoord.html}{\texttt{SkyCoord}}\xspace}
\newcommand{\astropyQuantity}{\href{http://docs.astropy.org/en/stable/api/astropy.units.Quantity.html}{\texttt{Quantity}}\xspace}
\newcommand{\astropyTime}{\href{http://docs.astropy.org/en/stable/api/astropy.time.Time.html}{\texttt{Time}}\xspace}
\newcommand{\astropyTable}{\href{http://docs.astropy.org/en/stable/api/astropy.table.Table.html}{\texttt{Table}}\xspace}

\newcommand{\secauthor}[1]{{\color{blue}Author:~\textit{#1}}}
\newcommand{\secunfilled}{{\color{red}Author:~\textit{Looking for volunteers!}}}

\begin{document}

\draft{\today}

\title{The Astropy Project: \\
       Sustaining and Growing a Community-oriented Open-source Project and
       the Latest Major Release (v5.0) of the Core Package}

\correspondingauthor{Astropy Coordination Committee}
\email{coordinators@astropy.org}

\author{Astropy Collaboration}
\noaffiliation
{\let\thefootnote\relax\footnote{{The author list has two parts: the authors that made significant contributions to the writing of the paper in order of contribution, followed by contributors to the \astropy Project in alphabetical order. \textbf{The position in the author list does not correspond to contributions to the \astropy Project as a whole.} A more complete list of contributors to the core package can be found in the \href{https://github.com/astropy/astropy/graphs/contributors}{package repository}, and at the \href{http://www.astropy.org/team.html}{\astropy team webpage}.}}}

% \input{author.tex}

\begin{abstract}
The \astropy Project supports and fosters the development of open-source and openly-developed
\python packages that provide commonly-needed functionality to the astronomical
community.
A key element of the \astropy Project is the core package \astropypkg, which serves as the
foundation for more specialized projects and packages.
In this article, we summarize key features in the core package as of the recent major
release, version 5.0, and provide major updates for the project.
We then discuss supporting a broader ecosystem of inter-operable packages,
including connections with several astronomical observatories and missions.
We also revisit the future outlook of the \astropy Project and the current
status of Learn Astropy.
We conclude by raising and discussing the current and future challenges facing the project.
\end{abstract}

\keywords{%
    Astrophysics - Instrumentation and Methods for Astrophysics
    ---
    methods: data analysis
    ---
    methods: miscellaneous
}


\section{Introduction} \label{sec:intro}

\secauthor{Adrian Price-Whelan}

The \python programming language is a high-level, interpreted (as opposed to
compiled) programming language that has become an industry standard across many
computational domains, technological sectors, and fields of research.
Despite claims to the contrary \citep{Portegies-Zwart:2020}, \python enables
scalable, time- and energy-efficient code execution \citep[e.g.,][]{Augier:2021}
with a focus on code readability, ease of use, and interoperability with other
languages.
Over the last decade, \python has grown enormously in popularity to become a
dominant programming language in the astronomical and broader scientific
communities.
For example, Figure~\ref{fig:python-mentions} shows the number of yearly
full-text mentions of \python as compared to a few other programming languages
in refereed articles in the astronomical literature, demonstrating its nearly
exponential growth in popularity.
The rapid adoption of \python by astronomy researchers, students, observatories,
and technical staff combined with an associated increase in awareness and
interest about open-source software tools is contributing to a paradigm shift in
the way research is done, data is analyzed, and results are shared in astronomy
and beyond.

One of the factors that has led to its rapid ascent in popularity in scientific
contexts has been the significant, volunteer-driven effort behind developing
community-oriented open-source software tools and fostering communities of users
and developers that have grown around these efforts.
Today, a broad and feature-diverse ``ecosystem'' of packages exists in the
\python scientific computing landscape: Roughly ordered from general-use to
domain-specific, this landscape now includes packages that provide core
numerical analysis functionality like \package{numpy} \citep{numpy:nature} and
\package{scipy} \citep{scipy}, visualization frameworks like
\package{matplotlib} \citep{matplotlib}, machine learning and data analysis
packages like \package{tensorflow} \citep{tensorflow}, \package{pymc3}
\citep{Salvatier:2016}, and \package{emcee} \citep{emcee}, domain-specific
libraries like \package{yt} \citep{yt:2011}, \package{plasmapy}
\citep{plasmapy}, \package{sunpy} \citep{sunpy:apj}, \package{Biopython}
\citep{biopython}, and \package{sympy} \citep{sympy} (to name a few in each
category).
The \astropypkg \citep{astropy:2013, astropy:2018} core package began in this
vein, as an effort to consolidate the development of commonly-used functionality
needed to perform astronomical research into a community-developed \python
package.

The \astropypkg core package was one of the first large, open-source \python
packages developed for astronomy and provides, among other things, software
functionality for reading and writing astronomy-specific data formats (e.g.,
FITS), transforming and representing astronomical coordinates, and representing
and propagating physical units in code.
An early description of the core functionality in \astropypkg can be found in
the first Astropy paper \citep{astropy:2013} or in detail in the core package
documentation.\footnote{\url{https://docs.astropy.org/}}
The \astropypkg core package is now largely stable in that the software
interface does not change without sufficient and significant motivation, and the
addition of new features into the core package has slowed significantly as
compared to the first years of its development.
This is largely driven by the fact that the core package now represents just one
piece of the broader astronomy \python context, and much new feature development
is now happening in more specialized packages that are expanding the
capabilities of the Astropy ecosystem by building on top of the foundations laid
by the \astropypkg core package.
Because of this natural expansion, the name Astropy has grown in scope beyond a
single \python library to become ``the Astropy Project.''

The Astropy Project is a community effort that represents the union of the
\astropypkg core package, the ecosystem of astronomy-specific software tools
that are interoperable with \astropypkg (Astropy Affiliated Packages),
\emph{and} the community of users, developers, and maintainers that participate
in Astropy efforts.
However, there is no institution responsible for managing the Astropy Project,
for funding or maintaining its development, or sustaining it into the future:
The Project is maintained and coordinated largely by volunteers.
While new Astropy-affiliated packages are being developed that expand upon the
core functionality in the \astropypkg package, representing a natural expansion
of the Astropy Project ecosystem, the needs of and challenges faced by the
Project are evolving.
In particular, the transition from focusing our energy on development and
maintenance of a single core package, to instead sustaining the core package and
fostering the development of the community and its expansion has been a key
issue faced by the Astropy Project in the last several years.

In this Article, we briefly describe recent key updates in the \astropypkg core
package since the last Astropy paper (``Paper II''; \citealt{astropy:2018}),
major updates in the governance, contributor base, and funding of the Project,
and discuss some of the future plans and challenges faced by the Astropy
Project.


\begin{figure}
    \begin{centering}
      % \includegraphics[width=\textwidth]{figures/python-mentions.pdf}
        \caption{
            Yearly full-text mentions of programming languages (indicated in the
            figure legend) in refereed publications in the astronomical
            literature database in the Astrophysics Data System (ADS;
            \citealt{ads}).
            \python has rapidly become the dominant programming language
            mentioned in refereed articles over the last 10 years.
        }
        \label{fig:python-mentions}
    \end{centering}
\end{figure}


\section{Major Updates to the Astropy Core Package} \label{sec:core-updates}

\subsection{New Long-term Support (LTS) Version: v5.0} \label{sec:core-v50}

\secauthor{Tom Robitaille}

Major versions of the core package -- that is to say versions which add and/or
modify functionality -- are released approximately every six months, and are
then maintained with releases that fix issues, until the next major version
is released. However, every two years a major release is designated as a
long-term support (LTS) release which continues to be maintained for up to two
years \citep{ape2}. The motivation for LTS releases is to provide longer-term stable
versions of astropy that users requiring a high level of stability can make
use of if they do not always need the latest features (this could include
for example telescope pipelines and so on). The 5.0 release of the core package was designated as LTS,
and since it was released at the end of 2021, it will be maintained until the
end of 2023.


\subsection{Highlighted Feature Development} \label{sec:core-features}

\secauthor{Nathaniel Starkman, Marten van Kerkwijk}

Summarize major pieces of development since the last paper (since v2.0). A few
ideas (taken from What's New pages) below, but feel free to expand on or remove
from this list:
\begin{itemize}
    \item Support for representing and transforming velocity data in coordinates, epoch propagation (v3.0)
    \item Improved support for astronomical time series: TimeSeries object (v3.2), Box Least Squares periodogram (v3.1)
    \item Overall improved support of Quantity throughout numpy (v4.0) and scipy
    \item Native support for Time, Quantity, and SkyCoord objects in Astropy tables
    \item APE 14 WCS
    \item A new SpectralCoord class for representing and transforming spectral quantities
    \item Support for dask/large arrays in several places in the core package (4.1, 5.0, and maybe other versions)
\end{itemize}


\section{Major Updates in the Astropy Project} \label{sec:project-updates}

\subsection{Project governance} \label{sec:project-governance}

\secauthor{Erik Tollerud}

%Briefly summarize new procedures and governance structure, new CoCo, election
%process overview, etc.

As part of the process of developing Astropy into a long-term sustainable
product, and to improve transparency and accountability, the Project agreed to
write down and formalize our governance structure (partly supported by explicit
funding for this purpose - see \S \ref{sec:project-funding}). At the 2019
Astropy Coordination Meeting, input was gathered from participants on what governance
structures existed in the associated Open Source Software communities, and what
would fit well with the needs of Astropy. This led into a ``retreat'' planned
for March 2020, but due to the COVID-19 pandemic, this became a series of
virtual meetings of the ``Astropy Governance Working Group''.  This group
drafted the APE 0 document \citep{ape0}, which was then eventually ratified and
implemented by the ``Astropy Governance Implementation Working Group'' in Fall
2021. While the process emphasized flexibility and the ability to adapt to
changing circumstances, it is expected that this is the framework Astropy's
governance will operate in for at least the medium-term future.

The APE0 \citep{ape0} document lays out the principles of this governance
structure, so we refer the reader to that document for a more thorough
description.  However here we highlight some key elements. While many of these
principles were already de facto true or have been discussed organically (and
have been discussed in earlier papers in this series), the APE0-based governance
aimed to provide a single place where the community can agree as a starting
point. With this in mind, it highlights the developer and user community of the
Astropy Project as the ultimate sources of authority, as well as the core
principle of ``do-ocracy'' that those who do work for the Project (be it coding,
training, or other less concrete contributions) gain more influence on the
outputs of the Project by virtue of their effort.  APE0 adds, however, the
concept of ``voting members'' - a self-governed part of that community who are
entrusted to elect the Coordination Committee. While this committee has existed
from the inception of the project, APE0 establishes a formal voting process for
this committee, and explicitly outlines the rights and responsibilities of the
Coordination Committee. This role is mainly to facilitate consensus and act as
the decision maker when other mechanisms have failed.  However, it also includes
powers that either require central authority or secrets (e.g., passwords), but
APE0 also charges the Committee to devolve responsibilities and seek community
input on these items as often as possible.

The first Coordination Committee election under these rules took place in Fall
2021, electing a mix of prior existing and new coordination committee members,
and was contested in the sense of more candidates than available slots. This
suggests the process is already working to serve the long-term interests of the
committee to both spread the coordination effort, and to ensure it is not
dominated by the same people for as long as the Project continues. While other,
more fine-grained governance improvements are planned for the future, it is
clear the foundation is now in place.


\subsection{Contributor base} \label{sec:project-contributors}

\secauthor{Adrian Price-Whelan}

Overview and statistics of contributors. Highlight changes since v2.0.

\begin{figure}
    \begin{centering}
      % \includegraphics{figures/contributor-summary.pdf}
        \caption{Placeholder figure!}
        \label{fig:contributor-summary}
    \end{centering}
\end{figure}

% Transition of needs: From new features, to community development, sustainability, ...
% - pipeline between User, Participant, Code contributor, Maintainer, Coordinator.

\subsection{Inclusion, Diversity, and Equity Programs} \label{sec:project-ide}

\secauthor{Lía Corrales}


\subsection{Funding} \label{sec:project-funding}

\secauthor{Aarya Patil}

Summarize funding sources (Moore, NASA) and amounts and what this has been used
for.


\section{Supporting the Ecosystem of Astronomical Python Software}
\label{sec:ecosystem}

\subsection{Community-oriented infrastructure}

\secauthor{Nicholas Earl}

The Astropy project supports the broader ecosystem by providing pre-configured
infrastructure packages that the community can use to support and maintain
their own software package infrastructure. These include tools to easily
generate documentation and setup automated testing, as well as provide
package scaffolding for new projects.

Sphinx is a common and useful tool for generating documentation for Python
packages. The Astropy project maintains a default Sphinx configuration along
with Astropy-specific extensions which can be easily added to community
projects via the
\href{https://github.com/astropy/sphinx-astropy}{\texttt{sphinx-astropy}} meta
package. This tool provides a pre-configured Sphinx setup compatible with
Astropy projects, which includes several extensions useful for generating API
documentation
(\href{https://github.com/astropy/sphinx-automodapi}{\texttt{sphinx-automodapi}}),
allowing for Numpy docstring parsing
(\href{https://github.com/numpy/numpydoc}{\texttt{numpydoc}}), embedded image
handling
(\href{https://github.com/sphinx-gallery/sphinx-gallery}{\texttt{sphinx-gallery}};
\href{https://github.com/python-pillow}{\texttt{pillow}}), advanced
documentation testing support
(\href{https://github.com/astropy/pytest-doctestplus}{\texttt{pytest-doctestplus}}),
and providing a custom documentation theme ideal for analysis packages
(\href{https://github.com/astropy/sphinx-astropy}{\texttt{astropy-sphinx-theme}}).

Community package testing infrastructure is supported through the
\href{https://github.com/astropy/pytest-astropy}{\texttt{pytest-astropy}}
meta-package, providing a unified testing framework with useful extensions
compatible with both Astropy- and non-Astropy-affiliated community packages.
This meta-package pulls in several
\href{https://github.com/pytest-dev/pytest}{\texttt{pytest}} plugins to help
with custom test headers
(\href{https://github.com/astropy/pytest-astropy-header}{\texttt{pytest-astropy-header}}),
accessing remotely-hosted data files in tests
(\href{https://github.com/astropy/pytest-remotedata}{\texttt{pytest-remotedata}}),
interoperability with documentation
(\href{https://github.com/astropy/pytest-doctestplus}{\texttt{pytest-doctestplus}}),
dangling file handle checking
(\href{https://github.com/astropy/pytest-openfiles}{\texttt{pytest-openfiles}}),
data array comparison support in tests
(\href{https://github.com/astropy/pytest-arraydiff}{\texttt{pytest-arraydiff}}),
sub-package command-line testing support
(\href{https://github.com/astropy/pytest-filter-subpackage}{\texttt{pytest-filter-subpackage}}),
improved mock object testing
(\href{https://github.com/pytest-dev/pytest-mock}{\texttt{pytest-mock}}), test
coverage reports and measurements
(\href{https://github.com/pytest-dev/pytest-cov}{\texttt{pytest-cov}}), and
configuring package for property-based testing
(\href{https://github.com/HypothesisWorks/hypothesis}{\texttt{hypothesis}}).

The Astropy Package Template helps facilitate the setup and creation of new
Python packages leveraging the Astropy ecosystem. This tool utilizes the
Cookiecutter project to walk users through the process of creating new
packages complete with documentation and testing support. Additionally, the
package template generation process includes the ability to setup
interoperability with GitHub, allowing for easy repository access from
documentation, as well as an example GitHub Actions workflow to demonstrate
the use of GitHub's continuous integration tooling.

\subsection{Affiliated packages}

\secauthor{Matt Craig, Brett Morris}

Highlight a few new affiliated packages and major updates to existing ones.
Include a big table of all affiliated packages and references (as in v2.0
paper).

All coordinated and affiliated packages are listed on the astropy
website\footnote{\url{https://www.astropy.org/affiliated}}. Since 
\cite{astropy:2018}, there has been an expansion of affiliated
packages for gravitational astrophysics including:
\texttt{PyCBC} for exploring gravitational wave signals, \texttt{lenstronomy} for
modeling strong gravitational lenses, \texttt{ligo.skymap} for visualizing
gravitational wave probability maps, and \texttt{EinsteinPy} for general
relativity and gravitational astronomy. There have been several packages
added to the ecosystem related to HEALPix: \texttt{astropy-healpix}
for a BSD-licensed HEALPix implementation, and \texttt{mocpy} for Multi-Order
Coverage maps. \texttt{astroalign} has been introduced for astrometric registration,
and \texttt{python-cpl} has been added for ESO pipelines and VLT data products. 
For ground-based astronomy, \texttt{baseband} has added IO capabilities for 
VLBI, and \texttt{SpectraPy} brings slit spectroscopy to the astropy ecosystem. 
We have \texttt{agnpy} for AGN jets, and \texttt{statmorph} for fitting galactic
morphological diagnostics. \texttt{saba} gives an interface to sherpa's fitting 
routines. For modeling interstellar dust extinction we have 
\texttt{dust\_extinction}, and for extracting features from time series we
have \texttt{feets}. We also have \texttt{corral} for managing data intensive parallel 
pipelines. \texttt{BayesicFitting} provides an interface for generic Bayesian 
inference, and \texttt{sbpy} \cite{Mommert2019} does calculations for asteroid and cometary
astrophysics. Finally, \texttt{synphot} provides an interface for synthetic photometry.

Several of the coordinated and affiliated packages described in \cite{astropy:2018} have had substantial improvements. \texttt{astroquery}
\cite{astroquery} has added
access to roughly a dozen new missions and data services, including JWST, and
the project has switched to a continuous release model. Every time a change
is committed to the main development branch it is published on the Python
Package Index (PyPI, ref) and available for installation. Formal releases are
still done a couple of times per year. \texttt{photutils} \cite
{photutils} released its first stable version, indicating that the API will
change less frequently, and there have been several significant performance
improvements.  \texttt{ccdproc} \cite{ccdproc} also released a new major
version, bringing better performance to some image combination operations.
The \texttt{regions} package \cite{pyregions}, for manipulating ds9-style
region definitions \cite{ds9}, added new ways to manipulate regions and
introduced new region types. \texttt{reproject}’s \cite{reproject} major new
feature is a function to align and co-add images to create a mosaic; better
support for parallelization was also added. \texttt{specutils}, the package
that defines containers for 1D and 2D spectra \cite{specutils}, also had its
first stable release and the addition of classes to read JWST data.
\texttt{stingray} has also had a major performance overhaul. The package
\texttt{gammapy} \cite{gammapy} also has major performance improvements,
has unified its API in preparation for its first stable release.

\subsection{Connections with data archives}

\secauthor{Adam Ginsburg}

Summary of Astroquery and engagement with archives. Cite astroquery paper.

\subsection{Connections with Observatories and Missions}

Brief summary of efforts in observatory- or mission-driven development that have
contributed to Astropy, and vice versa.

\subsubsection{James Webb Space Telescope}
\secauthor{Larry Bradley}

The James Webb Space Telescope (JWST) is a 6.5-meter space-based
infrared telescope that will provide unprecedented resolution and
sensitivity from 0.6 -- 28 microns. JWST will enable a broad range
of scientific investigations from exoplanets and their atmospheres
to the formation of galaxies in the very early universe. Its four
key scientific goals are to study the first light from stars and
galaxies, the assembly and evolution of galaxies, the birth of stars and
protoplanetary systems, and planetary systems and the origins of life.

The telescope launched on an Ariane 5 rocket on 2021 December 25 from
Kourou, French Guiana. After a series of successful deployments,
including the sunshield and primary and secondary mirrors, JWST reached
its orbit around the L2 Lagrange point on 2022 January 24. Commissioning
of the telescope optics and science instruments will occur from January
until the end of June 2022, when science operations are scheduled to
begin.

Software developers at the Space Telescope Science Institute (STScI),
the operations center of JWST, have been developing Python-based tools
for JWST since 2010 (starting with the JWST Calibration Reference Data
System) and have provided major contributions to \astropy from its
inception. The JWST instrument calibration pipelines, exposure-time
calculators, and data analysis tools are all written in Python and
depend on the \astropypkg core package and some coordinated and
affiliated packages. For the \astropypkg core package, the JWST mission
has provided extensive contributions to the \astropysubpkg{modeling},
\astropysubpkg{units}, \astropysubpkg{coordinates}, \astropysubpkg{wcs},
\astropysubpkg{io.fits}, \astropysubpkg{io.votable},
\astropysubpkg{stats}, \astropysubpkg{visualization}, and
\astropysubpkg{convolution} subpackages as well as to the general
package infrastructure and maintenance.

Likewise, JWST developers have provided significant contributions
to the \package{photutils} \citep{photutils}, \package{specutils}
\citep{specutils}, and \package{regions} \citep{regions} coordinated
packages and the \package{gwcs} \citep{gwcs} and \package{synphot}
\citep{synphot} affiliated packages. For example, development of the
\package{photutils} coordinated package for source detection and
photometry has largely been led by JWST contributions. JWST developers
have also made significant contributions to the \package{specutils}
coordinated package, which is used for analyzing spectroscopic data,
and the \package{regions} coordinated package, which is used to
handle geometric regions. The \package{gwcs} affiliated package for
generalized world coordinate systems was created specifically to handle
the complex world coordinate systems needed for JWST spectroscopic
data. The \package{synphot} affiliated package for synthetic photometry
was created at STScI and is a dependency of the JWST exposure-time
calculators. Further, the \package{ASDF} (Advanced Scientific Data
Format) package \citep{ASDF}, a next-generation interchange format for
scientific data, was initially developed at STScI to serialize JWST WCS
objects along with \astropypkg models, units, and coordinates. Over
its mission lifetime, JWST will continue its support in developing and
maintaining these critically dependent packages.


Gemini
\secunfilled

Cherenkov Telescope Array
\secauthor{Axel Donath, Maximilian Nöthe}

The Cherenkov Telescope (CTA) will be the next generation very-high-energy
gamma-ray observatory.
CTA will improve over the current generation of imaging atmospheric Cherenkov telescopes (IACTs)
by a factor of five to ten in sensitivity and will be able to observe the whole sky from a combination of two sites:
a northern site in La Palma, Spain, and a southern one in Paranal, Chile.
CTA will be able to observe gamma rays in a broad energy range from around $20\,\mathrm{GeV}$ to over $300\,\mathrm{TeV}$
using three different types of telescopes, in total over 100 telescopes are planned at the two sites.
CTA will also be the first open gamma-ray observatory.

The data analysis pipeline is developed as open source software and essentially split in two domains:
\begin{enumerate}
  \item In the low-level analysis, the properties of the recorded air-shower events
    have to be estimated from the raw data.
    The raw data consists of very short (\textasciitilde $40--100\,\mathrm{ns}$) videos recorded with the fast and
    sensitive cameras of the telescopes.
    This includes the energy, particle type and direction of origin of the particle that induced the air shower
    and the time the shower was recorded.
  \item In the higher-level analysis, these reconstructed event lists are used together with some
    characterization of the instrument response to perform the actual scientific analysis.
    This software will be delivered as CTA science tools to the future users of the Observatory.
\end{enumerate}

A prototype for the low-level analysis is \texttt{ctapipe} \citep{ctapipe},
a python package developed to perform all the necessary tasks to from the raw data
of Cherenkov telescopes to the reconstructed event lists.

The high-level analysis (or CTA science tools) will be based on the astropy affiliated package
Gammapy~\citep{gammapy}.

Both Gammapy and \texttt{ctapipe} are using astropy heavily, mainly for units, times, coordinate transformations,
tables and FITS IO.

As CTA will record gamma-ray events with a rate of up to $10\,000$ events per second,
it needs to perform a large number of coordinate transformations.
To enable this, CTA member M.~Nöthe contributed a major performance improvement
for large numbers of coordinates with different observation times,
based on earlier work by B.~Winkel.

Together with Gammapy maintainer A.~Donath and former maintainer C.~Deil,
a total of 107 merged pull requests were contributed to astropy.


Rubin Observatory
\secunfilled

LIGO/Virgo/KAGRA
\secauthor{Leo Singer}
\section{Future Plans for the Astropy Project} \label{sec:future}

\subsection{Roadmap}

\secauthor{Clara Brasseur}

Overview of roadmap and some highlights.

\section{Learn Astropy} \label{sec:learn}

\secauthor{Lía Corrales, David Shupe, Kelle Cruz + Learn team}

\subsubsection{Current status and scope}

{\it Learn Astropy} is an umbrella term that acknowledges the broad educational
efforts made by the Astropy Project, which are led by the Learn Team.
The efforts focus on developing online content and workshops covering
astro-specific coding tasks in Python.
As described in \citet{astropy:2018}, There are four different types of
Learn Astropy content: \textit{tutorials}, consisting of Jupyter Notebook
lessons that are published in HTML format online; \textit{guides}, which are a
series of lessons providing a foundational resource for performing certain
type of astronomical analyses; \textit{examples}, which are snippets of code
that showcase a short task that can be performed with Astropy or an affiliated
package; and \textit{documentation}, which is contained within the code base.
This categorization drives content development, infrastructure choices, and the
appearance of the Learn Astropy website.
The Learn Team meets weekly to work on creating, expanding, improving these
educational resources.

\begin{itemize}

\item {\bf Learn Astropy website:} The Learn Team re-launched the website in
2021 with a new infrastructure platform, built around full-text search and
interactive filtering functionality, with the goal of making content
discoverable as the Learn Astropy content catalog expands.
This work has been supported in part by the Dunlap grant.
We have adopted Algolia, a search-as-a-service cloud platform, to store the
full-text and metadata records of Learn Astropy's content.
The new Learn Astropy website is a JavaScript (Gatsby/React) application that
uses the Algolia service to power its search and filtering user interface.
Our Python-based application, Learn Astropy Librarian, populates data into the
Algolia service. We tuned the Librarian around specific content formats (such as
Jupyter Notebook-based tutorial pages and Jupyter Book-based guides) to more
accurately index content and heuristically extract metadata.
A consequence of the new platform is that we now maintain and compile content
separately from the website application itself, enabling new content types.
Tutorials, which as Jupyter Notebook-based, are now compiled into their own
Learn Astropy sub-site using nbcollection.
Guides, which are Jupyter Book-based, are also deployed as separate websites
using GitHub pages. This architecture opens future possibilities of indexing
third-party content, hosted elsewhere, such as on institutional websites.

\item \textbf{Tutorials} Currently, we have \# of tutorials, spanning a wide range of astronomical topics and common tasks.

\item \textbf{Guide} {\it ccdproc} Jupyter book has been completed.

\item {\bf Workshops:} the Project has been conducting workshops at winter
meetings of the American Astronomical Society since AAS 225 in January 2015.
Up to the start of the coronavirus pandemic, these were full-day in-person
workshops with as many as 90 participants and a dozen facilitators from the
project.
During the pandemic, these workshops were moved to an online format and split
into basic and advanced sessions.
Additionally, beginning with the AAS 238 online meeting, the workshops have been
expanded to Summer AAS meetings.
The Learn team finds that the workshop audience is best found as AAS meetings as
opposed to more general Python meetings.

The Astropy Project provided another mode of community engagement at AAS
Meetings 235 and 237 by organizing a NumFOCUS Sponsored Projects booth in the
AAS Exhibit Hall.
Funding for the exhibit hall was provided alternately by NumFOCUS and later by
the Moore Foundation funding (\textbf{check this}).
The booth hosted a series of Q\&A special sessions during AAS 235 and webinars
during the virtual AAS 237 meetings, to provide the general astronomy community
information and access to experts on a variety of open source astronomical
tools.
\end{itemize}

\subsubsection{Learn vision for the future}

The Learn plan going forward is to continue to improve the tutorial website and
facilate new content;
to index third-party tutorials; and to look for supportable opportunities to
expand the reach of Astropy workshops beyond the American Astronomical Society
meetings.



\subsection{Current and Future Challenges}

\section{Community Engagement}

\begin{itemize}
\item {\bf User forums:} The Astropy Project has historically maintained a number of avenues for users to seek help or gain access to the developer community. This includes the Astropy mailing list, the Python in Astronomy Facebook group, and the Astropy Slack workspace. The Moore Foundation grant allows this Slack space to
be on a paid plan; additionally NumFOCUS has negotiated a special rate for
open-source projets. While the Slack and Astropy-dev mailing lists are primarily used to discuss the project direction and updates, it was noted that the use of a Facebook community could present a barrier to open source. The Astropy Project identified a need for a public, archived, searchable, and easily-to-navigate interface for users to ask for help Accordingly, we have commissioned
a Discourse site which is more open than the Facebook group and more user-oriented
than Astropy Slack. A benefit of the Moore Foundation grant is that Astropy
developers are able to invoice as independent contractors the time they
spend helping users on these forums.
\end{itemize}
\secunfilled

\begin{itemize}
\item Attracting new contributors when the code has become quite complex,
\item Contributor to maintainer mentoring,
\item Long-term / sustained funding for maintaining infrastructure,
\item ...
\end{itemize}


\begin{acknowledgments}

We would like to thank the members of the community that have contributed to
\astropy, that have opened issues and provided feedback, and have supported the
project in a number of different ways.

The \astropy community is supported by and makes use
of a number of organizations and services outside the traditional
academic community. We thank Google for financing and organizing the
Google Summer of Code (GSoC) program, that has funded severals
students per year to work on \astropy related projects over the
summer. These students often turn into long-term contributors. We also
thank NumFOCUS and the Python Software Foundation for financial
support. Within the academic community, we thank
institutions that make it possible that astronomers and other developers on
their staff can contribute their time to the development of
\astropy projects.  We would like acknowledge the support of the
Space Telescope Science Institute, Harvard--Smithsonian Center for Astrophysics,
and the South African Astronomical Observatory.

Furthermore, the \astropy packages would not exist in their current form without
a number of web services for code hosting, continuous integration, and
documentation; in particular, \astropy heavily relies on GitHub, Travis CI,
Appveyor, CircleCI, and Read the Docs.

\astropypkg interfaces with the SIMBAD database, operated at CDS, Strasbourg,
France. It also makes use of the ERFA library \citep{erfa}, which in turn
derives from the IAU SOFA Collection\footnote{\url{http://www.iausofa.org}}
developed by the International Astronomical Union Standards of Fundamental
Astronomy \citep{sofa}.

\end{acknowledgments}

\software{\package{astropy} \citep{astropy:2013, astropy:2018},
          \package{numpy} \citep{numpy:nature},
          \package{scipy} \citep{scipy},
          \package{matplotlib} \citep{matplotlib},
          \package{Cython} \citep{cython},
          \package{photutils} \citep{photutils},
          \package{specutils} \citep{specutils},
          \package{regions} \citep{regions},
          \package{gwcs} \citep{gwcs},
          \package{synphot} \citep{synphot},
          \package{ASDF} \citep{ASDF}.
        %   \package{SOFA} \citep{sofa},
        %   \package{ERFA} (\citealt{erfa})
          }

\bibliographystyle{aasjournal}
\bibliography{refs}

% \appendix

% \section{List of Affiliated Packages}

% \begin{longrotatetable}
%     \begin{deluxetable*}{cccp{3in}c}
%     \tablecaption{Registry of affiliated packages.}
%     \label{tab:registry}
%     \tablehead{
%         \colhead{Package Name} &
%         \colhead{Stable} &
%         \colhead{PyPI Name} &
%         \colhead{Maintainer} &
%         \colhead{Citation}
%       }
%       \startdata
%         \input{registry.tex}
%       \enddata
%   \end{deluxetable*}
% \end{longrotatetable}


\end{document}
